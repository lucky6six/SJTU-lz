% !TEX root = ../main.tex

\chapter{实验结果与分析}


\section{实验环境与测试方法}
\subsection{实验环境}
\subsection{工作负载说明}

\begin{table}[!hpt]
    \caption[工作负载说明表]{工作负载说明表}
    \label{tab:firstone}
    \centering
    \begin{tabular}{@{} >{\raggedright\arraybackslash}p{0.41\linewidth}
                                            >{\raggedright\arraybackslash}p{0.36\linewidth}
                                            >{\raggedright\arraybackslash}p{0.30\linewidth} @{}}
        \toprule
        负载数据集 & 说明 & 参考链接 \\ \midrule
        tar、iso 等打包文件 &
            能识别文件中的模型数据部分 &
            全部认为是非模型文件 \\
        用户自定义非标准后缀的模型文件 &
            能够识别模型数据 &
            无效 \\
        Q4等没有压缩空间的高熵量化文件&
            可以将其识别为非模型数据 &
            将其作为模型数据进行编码,得到负收益。 \\
        header 等模型文件中的非模型部分 &
            可以将其识别为非模型数据 &
            将其作为模型数据进行编码,得到负收益。 \\
        集成侵入性 &
            可以在块处理阶段每块独立进行,对传统块级去重无影响,前序流程可以统一处理。 &
            集成困难 \\
        \bottomrule
    \end{tabular}
\end{table}

\subsection{测试对照方案}
\subsection{测试方法}

\section{各混合负载比例实验(去重率/时延)}

\section{各类型模型数据去重率实验}

\section{各类型非模型数据去重率实验}

\section{应用集成实验}

\section{其他实验}
\subsection{鉴别准确率}
\subsection{块大小影响实验}
\subsection{去重来源阶段实验}
\subsection{时延来源实验(性能开销分析)}

\section{本章小结}

