% !TEX root = ../main.tex

\chapter{总结与展望}


\section{全文总结}
本文聚焦云计算与人工智能融合背景下,云存储混合负载(传统非模型数据与 AI 模型数据共存)导致传统重复数据删除技术效能骤降的核心问题,系统性提出并实现了细粒度混合冗余去重系统(FHRD),为混合负载场景下的存储优化提供了完整解决方案。

针对传统去重技术对 AI 模型数据“识别失效”、对大块非模型数据“挖掘低效”的双重瓶颈,本文构建了“数据类型精准鉴别-差异化冗余提取”的分治架构。通过轻量级块级数据类型鉴别探针,基于字节粒度分组抽样熵值分析,实现了 98\% 以上的模型与非模型数据鉴别准确率,且吞吐量较比特级鉴别方法提升 69.6\%,有效规避了文件后缀元数据识别的高误报率问题。针对模型数据的浮点数指数位冗余特性,设计基于熵值结论的字节分组压缩策略,仅对低熵冗余组进行针对性处理,在保障数据完整性的前提下显著提升压缩效率;针对非模型数据的块内局部冗余,提出指数取整的双向子块定长切分方法,缓解传统切分的边界偏移问题,相似块检出率较传统方法提升 2-11 倍。

系统层面通过数据段聚合、容器化存储、多级索引缓存与流水线并行处理等优化手段,在控制额外计算开销的同时,确保了系统吞吐量与传统方案处于同一量级。实验验证表明,FHRD 在混合负载场景下较传统去重技术提升 20\%--30\% 的去重率,且模型数据占比越高、块体积越大,优势越显著,成功实现了“去重率提升”与“性能开销可控”的双重目标,验证了细粒度冗余识别与分治处理策略在混合负载存储优化中的有效性与实用性。
\section{未来工作展望}
尽管 FHRD 系统在混合负载去重中取得了预期效果,但结合数据存储技术的发展趋势与实际应用需求,未来可从以下方向进一步深化与拓展:

在技术优化层面,首先需提升数据类型鉴别的鲁棒性,针对日益多样化的 AI 模型格式(如自定义张量格式、量化模型文件)与混合编码数据,优化抽样策略与熵值特征提取方法,降低复杂场景下的误报与漏检率。

同时,借助 GPU、FPGA 等异构计算平台的并行处理能力,对字节分组压缩、子块切分等核心计算环节进行硬件加速,进一步缩短处理延迟,满足实时性存储场景需求。

在应用拓展层面,随着混合云与边缘计算的融合趋势加剧,需将 FHRD 系统适配分布式部署架构,优化跨节点数据索引同步与一致性维护机制,支持公有云、私有云与边缘节点的协同去重,以应对 PB 级海量数据的存储优化需求。